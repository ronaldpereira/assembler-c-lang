\documentclass{article}
\usepackage[utf8]{inputenc}

\title{Trabalho Prático 0 \\ Software Básico
\\ Assembler Simples}
\author{João Paulo Martino Bregunci - jpbregunci@gmail.com
\\ Ronald Davi Rodrigues Pereira - ronald.drp11@gmail.com}
\date{Outubro de 2016}

\usepackage{natbib}
\usepackage{graphicx}
\usepackage{indentfirst}

\begin{document}

\maketitle


\section{Introdução}
A tradução de cógido \textit{Assembly} para código de máquina é um dos problemas mais antigos da Ciência da Computação, dada a obvia dificuldade da programação diretamente em código de máquina. Desse problema, iniciou-se a proposta desse trabalho que será a construção de um \textit{Assembler} simples, o qual é capaz de gerar código executavel para a máquina \textit{Wombat2}. Este dado assembler em questão criado possui uma esquemática classica de \textit{2 passadas}, a fim de resolver qualquer tipo de referência posterior em instruções de desvio.
Ao final de todo o processo, cria-se um novo arquivo executável de saída que possa ser diretamente executado pela máquina. Vale ressaltar que todas as instruções de máquina possuem, nesta implementação, uma paridade com cada \textit{pseudo-instrução} de máquina.


\section{Explanação do Problema}
\subsection{Apresentação do Problema}
Como enunciado anteriormente na Introdução, o problema consiste na implementação de \textit{Assembler} simples. Sendo o maior problema relevante a tal a existência de referências futuras, ou seja, saltos na memória de instrução, cujo endereço ainda é desconheccido.

\subsection{Solução Generalizada do Problema}
Para resolver o prinicipal problema inerente, a existência de referências futuras, optou-se pela implementação de um \textit{Assembler de 2 passadas}, utilizando para tal uma \textit{Tabela de símbolos}, na qual armazena-se o endereço relativo de todos os procedimentos existentes no código, a fim de, na segunda passada, resolver tais referências futuras. A imagem a seguir auxília na compreensão do \textit{Assembler}:

\begin{figure}[h!]
\centering
\includegraphics[scale=0.4]{tp-1--1.png}
\caption{Fluxograma com os passos do \textit{Assembler}}
\label{fig:trieExample}
\end{figure}

O arquivo de entrada é representado acima pelo primeiro nodo e em cima de tal realiza-se a primeira passada por todo arquivo pelo \textit{Assembler}. Nessa passagem somente geram-se duas tabelas de referências, uma para os \textit{.data} e outra para a posição de todos os \textit{labels} no código, contando o PC \textit{program counter}, a partir do início do arquivo. Essa tabela é posteriormente utilizada para resolver todas as referências posteriores existentes, realizando assim a segunda passada. Ao final de todo esse processo é gerado o arquivo executável, que pode ser lido diretamente pela máquina.

\subsection{Implementação das Passagens}
Na primeira passagem há a criação de duas listas encadeadas, as quais são as duas tabelas de símbolos. Uma dessas tabelas contem todos os \textit{.data}, que serão prontamente substituidos e a outra tabela contem referências para os PC (\textit{program counter}) de labels encontradas na primeira passagem do \textit{Assembler}. Escolheram-se listas encadeadas para a estruturação da Tabela de Símbolos, pois não existem problemas eventuais na procura de elementos da lista em \textit{O(n)}, já que o número de \textit{labels} e \textit{.data} não prejudica severamente a performance do código.
Depois disso se realiza o processo corriqueiro de decodificação por meio de uma série de \textit{if, else if e else} para a leitura da instrução e a tradução do valores de eventuais registradores para os respectivos valores binários. Após todo esse processo se encontra o arquivo executável gerado.

\section{Casos de Teste}

Dois casos de teste foram formulados para o teste do montador, cada qual procurando demonstrar, testar e utilizar de comandos peculiares da arquitetura da máquina \textit{Wombat2}.

O primeiro programa realiza a busca do maior de três valores distintos e retorna tal valor ao quadrado acrescido de 1. Esse caso de teste é muito interessante, porque na implementação realizada utilizaram-se vários diferentes tipos de \textit{jump} no mesmo código, a fim de testar se realizava-se a montagem de maneira correta.

O segundo prorama emula muita das caracteristicas do primeiro, mas esse recebe 3 inteiros e retorna a média aritmetica de todos os elementos, sendo que elementos repetidos são contados somente uma vez. Nesse implementação a principal meta foi o teste se instrução como \textit{slt} e \textit{seq}, as quais possuem 3 operandos, realizam corretamente a decodificação.

No total entre os dois arquivos de teste foram testadas 18 instruções das 26 possíveis relativas a máquina \textit{Wombat2}. A codificação destas instruções para línguagem de máquina foi sucedida e externou o êxito do \textit{Assembler} aqui criado.

\subsection{}

\bibliographystyle{plain}
\bibliography{references}
\end{document}
